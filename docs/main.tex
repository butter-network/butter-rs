\documentclass{article}
\usepackage[utf8]{inputenc}

\title{Butter: An efficient decentralised platform}
\author{Alexandre Shinebourne}
\date{December 2021}

\begin{document}

\maketitle

\section{Introduction}
\section{Design}
\subsection{The platform vocabulary}
\begin{itemize}
    \item A 'server' becomes a 'listener'
    \item A 'client' becomes a 'caller'
    \item A node is an entity on the network that can both call and listen (expressed as a vertec)
    \item A peer is a node connected to other nodes on the network (expressed as a vertex with at least one connection)
    \item Peers are two or more connected nodes
    \item Each peer has a list of known peers
    \item A first-degree peer is one that a peer has in his list (expreesed as as edge)
\end{itemize}
\subsection{An informal analogy}
This is used to improve reasoning and creative thinking about decentralised systems.
\begin{itemize}
    \item A person can be thout of as a node (a unit of communication that can both listen and communicate information)
    \item A friend...
\end{itemize}
\end{document}
